%%%% CAPÍTULO 1 - INTRODUÇÃO
%%
%% Deve apresentar uma visão global da pesquisa, incluindo: breve histórico, importância e justificativa da escolha do tema,
%% delimitações do assunto, formulação de hipóteses e objetivos da pesquisa e estrutura do trabalho.

%% Título e rótulo de capítulo (rótulos não devem conter caracteres especiais, acentuados ou cedilha)
\chapter{Introdução}\label{cap:introducao}

Desde o nascimento, os seres humanos desenvolvem habilidades de reconhecimento 
e identificação de objetos. Logo na primeira semana de vida, os bebês estabelecem 
rapidamente reconhecimentos individuais, discriminando e demonstrando preferência 
pela face, voz e odor de sua própria mãe \cite{vieira2017}.

O termo biometria, do grego \textit{bios}-vida e \textit{metron}-medida, pode ser 
definida como ramo da ciência que estuda a identificação de aspectos físicos, biológicos e até 
comportamentais dos seres vivos. Na qual, são utilizados para distinguir indivíduos, 
a partir de suas características únicas \cite{ferreira2009}. Como por exemplo, a face, retina, 
íris, impressões digitais, geometria da mão, etc.

A biometria se tornou uma nova área de estudo a partir do antropologista francês
Alphonse Bertillon, em 1890, quando utilizou conceitos de biometria para a 
identificação de criminosos \cite{moraes2006}. 

Dentre as tecnologias atuais de segurança, a biometria tem sido amplamente 
utilizada, seja para acessar contas bancárias, aplicativos e até controlar 
o acesso a locais públicos e privados. Atualmente o reconhecimento facial 
é uma das biometrias mais estudadas, pois além da praticidade é considerada uma 
das formas mais seguras de identificação \cite{zhao2003}. 

Embora a identificação facial seja uma tarefa simples para os seres humanos, 
representa um desafio considerável para os computadores. Isso se deve em 
parte às restrições impostas pelo sistema biométrico facial, que abrangem 
o controle da iluminação e dos ângulos das imagens utilizadas. Além disso, 
várias variáveis estéticas, como barba, cabelo, uso de óculos e bonés, 
problemas na lente da câmera e até mesmo a possibilidade de inserção de 
dados incorretos, podem ocasionar falhas no processo de 
reconhecimento. \cite{cavalcanti2005}.

Assim, aprimorar a precisão de um sistema biométrico requer atenção 
durante o desenvolvimento, com foco na minimização de falsos 
positivos e falsos negativos no reconhecimento facial. Para superar esse 
desafio, é essencial encontrar uma abordagem que seja mais adequada ao 
sistema de autenticação por imagem. Entre as várias abordagens disponíveis, 
é fundamental avaliar a taxa de identificações incorretas e a taxa de 
casos não detectados. \cite{viola2004}.

% \begin{figure}[h!]
%     \centering
%     \caption{Imagens que lembram rostos humanos}
%     \includegraphics[scale=0.25]{figuras/pareidolia.jpg} 
%     \legend{Fonte: Adaptado de \citeonline{pareidoliaimg}.}
%     \label{fig:pareidolia}
%     \centering
% \end{figure}

Diante disso, o objetivo deste estudo é abordar o desenvolvimento de um 
protótipo que visa simplificar o controle de acesso através de um sistema 
de reconhecimento facial compacto e acessível, tornando-o de fácil 
utilização para qualquer pessoa. E também permitindo a integração 
com sistemas já existentes. 

\section{Objetivos}\label{sec:objetivos}

Nesta seção serão apresentados os objetivos deste trabalho e as etapas necessárias 
para o desenvolvimento do protótipo. Na qual, além da implementação do 
\textit{hardware}, também serão necessárias algumas etapas para a elaboração do \textit{software}, 
tendo como finalidade, obter uma alta assertividade no controle de acesso por 
reconhecimento facial.

\subsection{Objetivo geral}\label{subsec:objetivoGeral}

Este trabalho tem por objetivo realizar o estudo e desenvolvimento de um protótipo 
para controle de acesso por meio de reconhecimento facial. Para isso, 
serão utilizados algorítimos de processamento de imagens e aprendizagem 
profunda no microcontrolador ESP32-CAM.

\subsection{Objetivos específicos}\label{subsec:objetivosEspecificos}

Para que se cumpram os objetivos gerais, serão essenciais a realização de algumas etapas, 
as quais, são apresentadas a seguir:

\begin{itemize}
    \item  Desenvolver o \textit{hardware} para aquisição de imagens, levando em 
    consideração a luminosidade local e a qualidade da câmera. Garantindo assim, 
    bons resultados para a etapa de reconhecimento facial.
  
    \item Implementar um código que seja otimizado e organizado, o suficiente para 
    conseguir processar as imagens em tempo real. 
    
    \item Criar uma interface gráfica e física que permita aos usuários interagir 
    e utilizar de maneira intuitiva e simples.
    
    \item Por último, desenvolver o módulo de controle de acesso, que permitirá 
    ao administrador gerenciar e cadastrar novos usuários, concedendo ou 
    limitando o acesso de acordo com as necessidades.
\end{itemize}

\section{Justificativa}\label{sec:justificativa}

Os sistemas de reconhecimento facial foram uma grande solução durante a retomada 
das atividades presenciais após a pandemia do coronavírus, ajudando empresas
a promoverem uma maior segurança física, como também segurança sanitária, 
evitando contaminações e agilizando os processos. Ao contrário dos sistemas 
manuais, onde normalmente geram atrasos e demandam contato físico \cite{terra2020}.

Atualmente o controle de acesso mais comum, são aqueles que utilizam chaves 
e tags, porém, como possuem inúmeras fragilidades, estes procedimentos não são 
recomendados em locais que recebem um grande fluxo de pessoas, como por exemplo, 
hotéis e centros comerciais. Pois, desta forma, qualquer pessoa pode ter acesso, 
sem necessariamente estar credenciada.

Um aspecto relevante a considerar é que a implementação de sistemas 
automatizados tem o potencial de reduzir custos. Isso é notável, por 
exemplo, em condomínios e hotéis, onde tais sistemas podem desempenhar 
funções que, anteriormente, eram realizadas por porteiros e recepcionistas 
no cotidiano. Isso não só permite uma diminuição nas horas de trabalho 
desses profissionais, mas também resulta em economia de despesas para 
essas empresas.

Por fim, a facilidade desses sistemas, fazem com que os usuários não precisem 
mais memorizar senhas, ou carregar suas chaves, impactando positivamente na 
experiência de uso. Além disso, essa abordagem reduz significativamente a 
probabilidade de golpes e fraudes, uma vez que torna inviável o 
compartilhamento de acessos.
