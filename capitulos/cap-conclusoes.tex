%%%% CAPÍTULO 5 - CONCLUSÕES E PERSPECTIVAS
%%
\chapter{Conclusão}\label{cap:conclusoeseperspectivas}


O trabalho de desenvolvimento de um protótipo de reconhecimento 
facial alcançou todos os objetivos estabelecidos. O hardware foi 
projetado e construído com sucesso, utilizando componentes de baixo 
custo e disponíveis no mercado. O código foi implementado de 
forma otimizada e organizada, utilizando técnicas de aprendizado 
de máquina para realizar o reconhecimento facial. 

No que diz respeito à interface gráfica e física, a elaboração de 
telas específicas para a inicialização, login, cadastro e exclusão 
de usuários foi cuidadosamente projetada para fornecer uma 
experiência intuitiva ao usuário. 
Os fluxos de interação, incluindo o módulo do administrador com 
senhas, destacam a importância dos botões na interface física, 
desempenhando um papel significativo na navegação e inserção de 
valores.

Após a avaliação do protótipo, foram identificadas algumas 
possibilidades de melhoria. Uma delas seria aumentar o número máximo de 
usuários que podem ser cadastrados. Outra seria incluir IDs para cada 
usuário, o que permitiria excluí-los por meio desses identificadores. 
Também seria permitir que o administrador altere as senhas 
associadas ao protótipo.

Outra melhoria interessante seria a ativação do módulo Wi-Fi, o 
que permitiria a integração do protótipo com aplicativos ou 
serviços web. Por fim, uma melhoria de grande impacto envolveria 
a criação de uma placa personalizada, utilizando apenas o chip 
ESP32-S. Essa abordagem implicaria na liberação do I2C para uso 
geral, ampliando a disponibilidade de portas (GPIO) e 
possibilitando a incorporação de novos recursos.

No geral, o trabalho foi um sucesso e o protótipo desenvolvido 
apresenta um bom potencial de aplicação em diversos contextos. 
Com uma abordagem contínua de desenvolvimento e aprimoramento, esse sistema 
tem o potencial de se tornar uma ferramenta poderosa para controle de acesso 
em diversas aplicações.