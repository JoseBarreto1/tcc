%%%% RESUMO
%%
%% Apresentação concisa dos pontos relevantes de um texto, fornecendo uma visão rápida e clara do conteúdo e das conclusões do
%% trabalho.

\begin{resumoutfpr}%% Ambiente resumoutfpr

A biometria, em sistemas de segurança, tem sido amplamente utilizada devido sua 
confiabilidade, podendo ser empregada para acessar contas bancárias, autenticação de 
estabelecimentos, pagamento em dispositivos móveis, controle de acesso, etc. 
Dentre as biometrias disponíveis, o reconhecimento facial se destaca pela sua 
praticidade e por ser uma das biometrias mais estudadas e utilizadas. 
Podendo ser aplicada sem a necessidade de um contato físico. Diante disso, 
o presente estudo, visa o desenvolvimento de um protótipo de 
reconhecimento facial para controle e autenticação de acesso. Para essa finalidade, 
será utilizado um Single Board Computer (SBC), representado pelo dispositivo 
ESP32-CAM, que será responsável por capturar e classificar as imagens em tempo 
real. Sendo essas imagens processadas por intermédio de algorítimos de detecção facial 
da biblioteca Open Source Computer Vision Library (OPENCV) e de algorítimos 
de aprendizado de máquina.
\end{resumoutfpr}
