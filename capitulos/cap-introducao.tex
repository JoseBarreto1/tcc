%%%% CAPÍTULO 1 - INTRODUÇÃO
%%
%% Deve apresentar uma visão global da pesquisa, incluindo: breve histórico, importância e justificativa da escolha do tema,
%% delimitações do assunto, formulação de hipóteses e objetivos da pesquisa e estrutura do trabalho.

%% Título e rótulo de capítulo (rótulos não devem conter caracteres especiais, acentuados ou cedilha)
\chapter{Introdução}\label{cap:introducao}

Desde o nascimento o ser humano desenvolve sua capacidade de reconhecer características 
e essa capacidade atua como um dos principais fatores evolutivos da espécie humana. 
Sendo as primeiras demonstrações de reconhecimento de características por humanos, 
datados desde o período pré-histórico.

Biometria (do grego bios vida e metron medida) é a medição de aspectos físicos, 
biológicos e até comportamentais de seres vivos. Esse reconhecimento visa distinguir o indivíduo de 
outros no meio em que ele está inserido. A partir daí, é possível permitir ou negar determinada 
ação ou recurso a essa pessoa.

Dentre as tecnologias atuais de biometria, o reconhecimento facial é uma das mais estudadas 
atualmente, e, também um dos meios mais seguros de identificação e integridade dos 
sistemas que a utilizam, em além de oferecer no processo de autenticação, pois é 
instantâneo, não sendo necessário utilizar equipamentos especiais para identificação uma webcam ou 
câmera de celular é suficiente para extrair um recurso.

Embora promissora, a biometria enfrenta desafios que devem ser contrabalançados para 
não levar a interpretações e aplicações equivocadas. Dentre os principais desafios constam: 
excesso de informação, paradoxo da população, privacidade, intrusividade, ruído, vulnerabilidade 
e classificação.

Desta forma, observa-se que a biometria deve obstáculos para se tornar um padrão de 
segurança antifraude. O principal problema da detecção de rosto é determinar se a imagem 
arbitrária representa um rosto humano ou restaurar as coordenadas do rosto reconhecido.

Como solução, a literatura recente na área propõe técnicas de visão computacional e 
reconhecimento que, diante do progresso dos sistemas computacionais, permitem corredeiras, 
levando em consideração a relação de custo entre a velocidade da informação e a limitação de 
tempo.

Diante disso, o presente estudo visa discorrer sobre o desenvolvimento de um protótipo 
para o controle de acesso utilizando reconhecimento facial com SBC e OPENVC.

\section{Objetivos}\label{sec:objetivos}

Neste capítulo serão apresentados os objetivos desse trabalho, estes que são descrever
e analisar os métodos utilizados tanto no processamento de imagens, quanto no reconhecimento
facial. Utilizando métodos para desenvolver um algoritmo combinando técnicas de 
processamento de imagens e aprendizado de máquina, buscando o aperfeiçoamento na 
combinação das técnicas estudadas.

\subsection{Objetivo geral}\label{subsec:objetivoGeral}

Este trabalho tem como objetivo realizar o estudo, implementação e desenvolvimento 
de um dispositivo para controle de acesso por meio de reconhecimento facial, em um Single 
Board Computer (SBC), utilizando a biblioteca de visão computacional OpenCV (Open Source 
Computer Vision Library).

\subsection{Objetivos específicos}\label{subsec:objetivosEspecificos}

Para que se cumpram os objetivos gerais, são necessárias várias etapas, as quais são
apresentadas a seguir:

\begin{itemize}
    \item  Desenvolver o \textit{hardware} para aquisição de imagens, levando em 
    consideração a luminosidade do local e a qualidade da câmera, com o objetivo 
    de obter com uma qualidade razoável para a filtragem e processamento da imagem.
  
    \item Implementar um código que seja otimizado e organizado, o suficiente para 
    conseguir filtrar e processar as imagens em tempo real. 
    
    \item Desenvolver uma interface física, onde os usuários e utilizadores possam
    interagir e utilizar de forma simples e prática. Permitindo uma boa comunicação,
    de forma simples e direta. 
    
    \item Criar um sistema para controle de acesso, onde o usuário administrador
    irá gerenciar e ser responsável pelo cadastro dos demais utilizadores, seja 
    em um ambiente particular (casas e condomínios) corporativo (empresas) ou 
    publico (Escolas e Universidades) permitindo ou restringindo o acesso, 
    conforme julgar necessário. 
\end{itemize}

\section{Justificativa}\label{sec:justificativa}

Os sistemas de reconhecimento facial foi uma grande solução para a retomada 
das atividades presenciais após a pandemia do Coronavírus, ajudando empresas
a promoverem uma maior segurança física, como também segurança sanitária, 
evitando contaminações e agilizando os processos. Pois ao invés de 
um sistema manual, que normalmente geram atrasos, com a aplicação dessa tecnologia, 
é possível utilizar mais de um aparelho, aumentando a eficiência e rapidez 
no controle de acesso.

O grande risco do uso de chaves e tags, é que qualquer pessoa pode entrar 
sem necessariamente estar credenciado para aquele local. 
Sendo esta a fragilidade de locais que recebem um grande fluxo de pessoas, 
porém não possuem sistemas para controlar e acompanhar as entradas 
e saídas.

Um outro ponto importante, é a economia que este tipo de sistema pode gerar. 
Com o controle de acesso automatizado, locais como condomínios, hotéis 
ou recepções comerciais, passam a não ter mais necessidade de porteiros em tempo 
integral, desta forma, reduzindo custos aos que o utilizam. 

Com os leitores faciais, além do acesso mais seguro e exclusivo aos usuários 
credenciados, por exemplo, moradores, proprietários e funcionários, também
é possível credenciar usuários de forma temporária, limitando e liberando 
os acessos, dentro do tempo que for necessário.

\section{Cronograma}\label{sec:cronograma}

O cronograma abaixo mostra qual será a sequência de pesquisa de
desenvolvimento do trabalho. como no exemplo \autoref{quad:quadro1}

\begin{tabframed}[htb]
    \caption{Cronograma de desenvolvimento do trabalho}
    \label{quad:quadro1}
    \begin{tabular}{|l|*{12}{p{0.04\textwidth-\columnsep}|}}
    \hline
    \multirow{2}{*}{\textbf{Etapa}} & \multicolumn{5}{c|}{\textbf{2022}} & \multicolumn{6}{c|}{\textbf{2023}} \\  
    \cline{2-12}
    &\textbf{Ag} &\textbf{Se} &\textbf{Ou} &\textbf{No} &\textbf{De} &\textbf{Ja} &\textbf{Fe} &\textbf{Ma} &\textbf{Ab} &\textbf{Ma} &\textbf{Ju} \\ \hline
    Definição do tema                                 &X & & & & & & & & & & \\   \hline
    Definição do escopo do trabalho                   &X & & & & & & & & & & \\   \hline
    Analise de viabilidade e planejamento da execução &X & & & & & & & & & & \\   \hline
    Revisão Bibliográfica                             & &X &X &X & & & & & & & \\ \hline
    Pesquisa sobre Esp32 e OpenCV                     & &X &X & & & & & & & & \\  \hline
    Estudo da plataforma Nextion Editor               & & & &X &X & & & & & & \\  \hline
    Desenvolvimento do protótipo                      & & & & &X &X &X & & & & \\ \hline
    Aplicação do reconhecimento facial                & & & & & &X &X & & & & \\  \hline
    Análise dos resultados                            & & & & & & &X &X & & & \\  \hline
    Correção de erros e aprimoramentos                & & & & & & & &X &X &X & \\ \hline
    Escrita do trabalho                               &X &X &X &X &X &X &X &X &X &X &X \\ \hline
    Apresentação do trabalho                          & & & & & & & & & & &X \\   \hline
    \end{tabular}
    \fonte{}%% Fonte
\end{tabframed}
