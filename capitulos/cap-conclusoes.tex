%%%% CAPÍTULO 5 - CONCLUSÕES E PERSPECTIVAS
%%
\chapter{Conclusão}\label{cap:conclusoeseperspectivas}

Diante do exposto, pode-se notar que o avanço da tecnologia em diversos campos 
tem possibilitado à sociedade moderna oferecer as mais diversas facilidades aos 
seus indivíduos. Hoje, é possível realizar transações financeiras de casa, 
realizar reuniões com pessoas que estão a milhares de quilômetros de distância, 
participar de cursos e conferências ministradas em outro país para viajar de um 
continente para outro em poucas horas.

No entanto, todas essas conveniências, e um número crescente de quem tira proveito 
delas, tornou essencial o uso de mecanismos pessoais cada vez mais robustos que 
podem provar que um indivíduo é quem eles afirmam ser. Esses mecanismos, que são 
na forma de cartões magnéticos, senhas pessoais, carteiras de identidade, 
passaportes, entre outros, e, também trazem uma série de problemas associados, 
como perda, adulteração, empréstimo e dificuldade em ou armazenamento de vários 
códigos, entre outros.

O processo de identificação pessoal baseado na biometria tenta minimizar esses 
problemas, pois deixa de ser baseado em algo que o indivíduo possui", ou "algo que 
o indivíduo passa a considerar o próprio indivíduo como código de identificação.

Assim, o controle de acesso baseado em funcionalidades vem se mostrando uma área 
extremamente atrativa para explorar e experimentar novas abordagens, visto que 
possui demanda crescente e extremamente rica em termos de abordagens e técnicas a 
serem implementadas.
