%%%% utfprpb-dados.tex, 2019/12/03
%%%% Copyright (C) 2020 Vinicius Pegorini (vinicius@utfpr.edu.br)
%%
%% This work may be distributed and/or modified under the conditions of the
%% LaTeX Project Public License, either version 1.3 of this license or (at your
%% option) any later version.
%% The latest version of this license is in
%%   http://www.latex-project.org/lppl.txt
%% and version 1.3 or later is part of all distributions of LaTeX version
%% 2005/12/01 or later.
%%
%% This work has the LPPL maintenance status `maintained'.
%%
%% The Current Maintainer of this work is Vinicius Pegorini.
%% Updated by:
%% - Marco Aurélio Graciotto Silva;
%% - Rogério Aparecido Gonçalvez;
%% - Luiz Arthur Feitosa dos Santos.
%%
%% This work consists of the files utfpr.cls, main.tex, and
%% variaveis.tex.
%%
%% A brief description of this work is in readme.txt.

%% Documento
%% Luiz: Define a fonte do texto da monografia
\fonteTexto{\sfdefault} % utilize \rmdefault para Times New Roman ou \sfdefault para Arial
\TipoDeDocumento{Trabalho de Conclusão de Curso de Graduação}%% Tipo de documento: "Tese", "Dissertação" ou "Trabalho de Conclusão de Curso de Graduação", "Estágio Supervisionado"
\NivelDeFormacao{Bacharelado}%% Nível de formação: "Doutorado", "Mestrado", "Bacharelado" ou "Tecnólogo" - ATENÇÃO, isso será utilizado para alterar a formatação do trabalho, pois pode haver formatações distintas dependendo o nível/tipo de trabalho.


%% luiz
% Template LaTex criado pelo Departamento Acadêmico de Computação (DACOM)
% da Universidade Tecnológica Federal do Paraná - Campus Campo Mourão (UTFPR-CM)
% Criado e alterado pelos professores:
% - Marco Aurélio Graciotto Silva
% - Rogério Aparecido Gonçalvez
% - Luiz Arthur Feitosa dos Santos
% Esse template utiliza a licença CC BY:
% Esta licença permite que outros distribuam, remixem, adaptem e criem a partir deste trabalho, mesmo para fins comerciais, desde que atribuam o devido crédito pela criação original.
% https://creativecommons.org/licenses/by/4.0/deed.pt_BR

% Dados do curso. Caso seja BCC:
\program{}
\programen{Undergradute Program in Computer Science}
\degree{Bacharel}
\degreearea{Engenharia Eletrônica}
% Caso seja TSI:
% \program{Curso Superior de Tecnologia em Sistemas para Internet}
% \programen{Undergradute Program in Tecnology for Internet Systems}
% \degree{Tecnólogo}
% \degreearea{Tecnologia em Sistemas para Internet}


% Dados da disciplina. Escolha uma das opções e a descomente:
% TCC1:
%\goal{Proposta de Trabalho de Conclusão de Curso de Graduação}
%\course{Trabalho de Conclusão de Curso 1}
% TCC2:
 \goal{Trabalho de Conclusão de Curso de Graduação}
 \course{Trabalho de Conclusão de Curso 2}


% Dados do TCC (precisa alterar)
\author{JOSÉ BARRETO DOS SANTOS JUNIOR}  % Seu nome
\authorbib{Silva, João da} % Seu nome para referência bibliográfica (Sobrenome, Nome)
\title{DESENVOLVIMENTO DE PROTÓTIPO PARA CONTROLE DE ACESSO UTILIZANDO RECONHECIMENTO FACIAL} % Título do trabalho
\titleen{Development of a prototype for access control using facial recognition} % Título traduzido para inglês
\advisor{Eduardo Giometti Bertogna} % Nome do orientador. Lembre-se de prefixar com "Prof. Dr.", "Profª. Drª.", "Prof. Me." ou "Profª. Me."}
% Se não houver corientador, comente a linha a baixo
%\coadvisor{} % Nome do coorientador, caso exista. Caso não exista, comente a linha.
\depositshortdate{2023} % Ano em que depositou este documento
\approvaldate{24/novembro/2023}

% Dados do curso que não precisam de alteração
\university{Universidade Tecnológica Federal do Paraná}
\universityen{Federal University of Technology -- Paraná}
\universitycampus{Campus Campo Mourão}
\universityunit{Departamento Acadêmico de Computação}
\address{CAMPO MOURÃO}
\addressen{Campo Mourão, PR, Brazil}
\documenttype{Monografia}
\documenttypeen{Monograph}
\degreetype{Graduação}

\evalboardmember{Prof. Dr. Marcio Rodrigues da Cunha}{Avaliador(a) 1}{UTFPR}
\evalboardmember{Prof. Dr. Osmar Tormena Junior}{Avaliador(a) 2}{UTFPR}
\evalboardmember{Prof. Dr. Eduardo Giometti Bertogna}{Orientador(a)}{UTFPR}
%\evalboardmember{Nome completo e por extenso do Membro 4}{Título (especialização, mestrado, doutorado}{Nome completo e por extenso da instituição a qual possui vínculo}

%% Palavras-chave e keywords
%% ATENÇÃO - você deve indicar a quantidade de palavras chaves para o template LaTeX utilizar o pontuação correta!
\NumeroDePalavrasChave{3}%% Número de palavras-chave (máximo 5)
\PalavraChaveA{Reconhecimento facial}%% Palavra-chave A
\PalavraChaveB{Controle de acesso}%% Palavra-chave B
\PalavraChaveC{Biometria}%% Palavra-chave C
%\PalavraChaveD{Palavra-chave 4}%% Palavra-chave D
%\PalavraChaveE{Palavra-chave 5}%% Palavra-chave E

%% ATENÇÃO - você deve indicar a quantidade de keywords para o template LaTeX utilizar o pontuação correta!
\NumeroDeKeywords{3}%% Número de keywords (máximo 5)
\KeywordA{Facial recognition}%% Keyword A
\KeywordB{Access control}%% Keyword B
\KeywordC{Biometry}%% Keyword C
%\KeywordD{Keyword 4}%% Keyword D
%\KeywordE{Keyword 5}%% Keyword E


% É obrigatório o uso de uma licença Creative Commons (CC) nos trabalhos de TCC pelos cursos ligados a DACOM da UTFPR-CM.
% Veja: http://portal.utfpr.edu.br/biblioteca/trabalhos-academicos/docentes/procedimento-de-entrega-graduacao

% Sendo assim, escolha com o seu orientador uma das licenças CC a seguir: 

% CC BY: Esta licença permite que outros distribuam, remixem, adaptem e criem a partir deste trabalho, mesmo para fins comerciais, desde que atribuam o devido crédito pela criação original. Essa é a menos restritiva.
\licenca{ccby}

% CC BY CA: Esta licença permite que outros remixem, adaptem e criem a partir deste trabalho, mesmo para fins comerciais, desde que atribuam o devido crédito e que licenciem as novas criações sob termos idênticos.
%\licenca{ccbysa}

% CC BY ND: Esta licença permite a redistribuição, comercial e não comercial, desde que o trabalho seja distribuído inalterado e no seu todo, com crédito ao autor.
%\licenca{ccbynd}

% CC BY NC: Esta licença permite que outros remixem, adaptem e criem a partir deste trabalho para fins não comerciais, e embora os novos trabalhos tenham de atribuir o devido crédito e não possam ser usados para fins comerciais, os trabalhos derivados não têm que serem licenciados sob os mesmos termos.
%\licenca{ccbync}

% CC BY NC SA: Esta licença permite que outros remixem, adaptem e criem a partir deste trabalho para fins não comerciais, desde que atribuam ao autor o devido crédito e que licenciem as novas criações sob termos idênticos.
%\licenca{ccbyncsa}

% CC BY NC ND: Esta licença só permite que outros façam download do trabalho e o compartilhe desde que atribuam crédito ao autor, mas sem que possam alterá-los de nenhuma forma ou utilizá-los para fins comerciais. Essa é a mais restritiva.
%\licenca{ccbyncnd}

% Deixar sem licença - isso é aplicado apenas aos trabalhos que não são obrigados a ter licença. Na duvida verifique isso com o seu orientador e professor responsável pelo TCC. Para deixar o texto sem licença deixe o comando licença em brando ou deixe comentado.
%\licenca{}
% by DACOM/UTFPR-CM