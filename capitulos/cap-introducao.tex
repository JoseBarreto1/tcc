%%%% CAPÍTULO 1 - INTRODUÇÃO
%%
%% Deve apresentar uma visão global da pesquisa, incluindo: breve histórico, importância e justificativa da escolha do tema,
%% delimitações do assunto, formulação de hipóteses e objetivos da pesquisa e estrutura do trabalho.

%% Título e rótulo de capítulo (rótulos não devem conter caracteres especiais, acentuados ou cedilha)
\chapter{Introdução}\label{cap:introducao}

Desde o nascimento o ser humano desenvolve sua capacidade de reconhecer características 
e essa capacidade atua como um dos principais fatores evolutivos da espécie humana. 
Sendo as primeiras demonstrações de reconhecimento de características por humanos, 
datados desde o período pré-histórico.

Biometria (do grego bios vida e metron medida) é a medição de aspectos físicos, 
biológicos e até comportamentais de seres vivos. Esse reconhecimento visa distinguir o indivíduo de 
outros no meio em que ele está inserido. A partir daí, é possível permitir ou negar determinada 
ação ou recurso a essa pessoa.

Dentre as tecnologias atuais de biometria, o reconhecimento facial é uma das mais estudadas 
atualmente, e, também um dos meios mais seguros de identificação e integridade dos 
sistemas que a utilizam, em além de oferecer no processo de autenticação, pois é 
instantâneo, não sendo necessário utilizar equipamentos especiais para identificação uma webcam ou 
câmera de celular é suficiente para extrair um recurso.

Embora promissora, a biometria enfrenta desafios que devem ser contrabalançados para 
não levar a interpretações e aplicações equivocadas. Dentre os principais desafios constam: 
excesso de informação, paradoxo da população, privacidade, intrusividade, ruído, vulnerabilidade 
e classificação.

Desta forma, observa-se que a biometria deve obstáculos para se tornar um padrão de 
segurança antifraude. O principal problema da detecção de rosto é determinar se a imagem 
arbitrária representa um rosto humano ou restaurar as coordenadas do rosto reconhecido.

Como solução, a literatura recente na área propõe técnicas de visão computacional e 
reconhecimento que, diante do progresso dos sistemas computacionais, permitem corredeiras, 
levando em consideração a relação de custo entre a velocidade da informação e a limitação de 
tempo.

Diante disso, o presente estudo visa discorrer sobre o desenvolvimento de um protótipo 
para o controle de acesso utilizando reconhecimento facial com SBC e OPENVC.

\section{Objetivos}\label{sec:objetivos}

Este trabalho tem como objetivo realizar o estudo, implementação e desenvolvimento 
de um dispositivo para controle de acesso por meio de reconhecimento facial, em um Single 
Board Computer (SBC) e utilizando a biblioteca de visão computacional OpenCV (Open Source 
Computer Vision Library).

\subsection{Objetivo geral}\label{subsec:objetivoGeral}

O objetivo geral se refere ao resultado do trabalho realizado, enfatizando o que esse 
trabalho deixa para a comunidade acadêmica, para a sociedade e/ou para o ambiente profissional. 
Deve ser apresentado de forma a abranger o resultado principal do teste.

O objetivo geral e os específicos devem iniciar com verbo. Sugere-se que o objetivo geral 
contenha no máximo 3 (três) linhas, conforme exemplo abaixo:

Desenvolver um protótipo de um sistema de software para determinar a capacidade 
produtiva de pequenas empresas com base em estudos de cronoanálise industrial para pequenas 
empresas com produção em série.

\subsection{Objetivos específicos}\label{subsec:objetivosEspecificos}

Os objetivos específicos são opcionais, ou seja, somente devem ser apresentados...

\section{Justificativa}\label{sec:justificativa}

Justificar o objeto de...

\section{Estrutura do trabalho}\label{sec:estruturaTrabalho}

A estrutura do trabalho contém uma relação dos capítulos e uma descrição...

\section{Cronograma}\label{sec:cronograma}

O cronograma abaixo mostra qual será a sequência de pesquisa de
desenvolvimento do trabalho.

