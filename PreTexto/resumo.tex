%%%% RESUMO
%%
%% Apresentação concisa dos pontos relevantes de um texto, fornecendo uma visão rápida e clara do conteúdo e das conclusões do
%% trabalho.

\begin{resumoutfpr}%% Ambiente resumoutfpr
	A biometria, em sistemas de segurança, tem sido amplamente utilizada devido sua 
	confiabilidade, sendo empregada para acessar contas bancárias, autenticação de 
	estabelecimentos, pagamento em dispositivos móveis, controle de acesso, etc. 
	Dentre as biometrias disponíveis, o reconhecimento facial se destaca pela sua 
	praticidade e por ser uma das biometrias mais estudadas e utilizadas, 
	podendo ser aplicada sem a necessidade de um contato físico. Diante desse contexto,  
	o presente estudo tem como objetivo desenvolver um protótipo de reconhecimento facial 
	para controle de acesso. Nesse processo, empregou-se uma placa ESP32-CAM para 
	realizar o reconhecimento facial, em conjunto com um \textit{display} LCD para simplificar 
	a interação do usuário com o protótipo.  Em termos gerais, o protótipo atendeu às 
	expectativas, proporcionando a criação de um \textit{hardware} compacto e de fácil utilização, 
	permitindo a autenticação de usuários em tempo real.
\end{resumoutfpr}
	