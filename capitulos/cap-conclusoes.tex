%%%% CAPÍTULO 5 - CONCLUSÕES E PERSPECTIVAS
%%
\chapter{Conclusão}\label{cap:conclusoeseperspectivas}

O trabalho, de forma geral, apresentou bons resultados. O desenvolvimento e 
implementação do protótipo de controle de acesso por 
reconhecimento facial utilizando o ESP32-CAM mostraram-se promissores, atendendo 
aos requisitos do projeto. Desta forma, o sistema demonstrou 
a capacidade de detectar e reconhecer rostos, além de executar com sucesso 
tarefas como cadastro de usuários e a implementação de um temporizador.

No entanto, algumas limitações foram identificadas durante a elaboração do 
protótipo. Notavelmente, a falta de recursos de processamento do ESP32 limitou 
a plena capacidade de processamento de imagem e visão computacional. 
A ausência de bibliotecas de visão computacional amplamente disponíveis 
para o ESP32 também representou um desafio. No entanto, soluções alternativas 
foram aplicadas, permitindo o reconhecimento facial com relativo sucesso.

Para aprimorar ainda mais o sistema, seria essencial a inclusão de bibliotecas 
de visão computacional mais abrangentes, como o OpenCV, que ofereceriam mais 
opções e métodos de processamento de imagem. Além disso, a avaliação da 
influência da luminosidade na precisão do reconhecimento é um tópico importante 
a ser investigado, bem como a otimização dos parâmetros para melhorar a taxa 
de assertividade.

Apesar das limitações, o protótipo atendeu a praticamente todos os requisitos 
do projeto e mostrou-se uma base sólida para desenvolvimentos futuros. A 
implementação de melhorias, como a possibilidade de configurar o número de 
usuários cadastrados, exibir o ID do usuário e permitir a exclusão de usuários 
por ID, tornaria o sistema ainda mais flexível e prático. 

Além disso, a perspectiva de integrar o protótipo em aplicativos e servidores 
web expandiria suas aplicações práticas, tornando-o útil em sistemas de controle 
de acesso online.

Conforme mencionado anteriormente, uma melhoria substancial poderia ser alcançada 
ao desenvolver uma placa personalizada, utilizando apenas o chip ESP32. Isso permitiria 
uma maior flexibilidade na alocação de recursos, liberando GPIOs adicionais 
para integração de novos recursos e funcionalidades.

Em resumo, o protótipo de controle de acesso por reconhecimento facial 
demonstrou ser uma solução viável e suas limitações podem ser superadas 
com melhorias no \textit{hardware}, \textit{software} e integração em sistemas mais amplos. 
Com uma abordagem contínua de desenvolvimento e aprimoramento, esse sistema 
tem o potencial de se tornar uma ferramenta poderosa para controle de acesso 
em diversas aplicações.